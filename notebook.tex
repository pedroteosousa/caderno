\documentclass{article}

\title{ICPC Notebook}
\author{pedroteosousa }
\date{}

\usepackage[utf8]{inputenc}
\usepackage[hidelinks]{hyperref}
\usepackage{listings}
\usepackage{xcolor}
\usepackage{multicol}
\usepackage[a4paper, margin=0.5in]{geometry}

\lstset{
	tabsize=4,
	showstringspaces=false,
	commentstyle=\color{black},
	keywordstyle=\color{blue},
	stringstyle=\color{red}
}

\begin{document}
%\begin{multicols}{2}

\maketitle
\tableofcontents

\section{Geometry}
\subsection{Miscellaneous Geometry}
\begin{lstlisting}[language=C++]
double EPS = 1e-12;

struct point {
	double x, y;

	point () {}
	point (double a = 0, double b = 0) { x = a; y = b; }
	point (const point &p) { x = p.x; y = p.y; }

	point operator+ (const point &p) { return {x+p.x, y+p.y}; }
	point operator- (const point &p) { return {x-p.x, y-p.y}; }
	point operator* (double c) { return {c*x, c*y}; }
	point operator/ (double c) { return {x/c, y/c}; }

	double operator^ (const point &p) { return x*p.y - y*p.x; }
	double operator* (const point &p) { return x*p.x + y*p.y; }

	point rotate (double c, double s) {
		return {x*c - y*s, x*s + y*c}; 
	}
	point rotate (double ang) {
		return rotate(cos(ang), sin(ang));
	}

	double len() { return hypot(x, y); }

	bool operator< (const point &p) const {
		return (x < p.x) || (x == p.x && y < p.y);
	}
};

double side(point a, point b, point c) {
	return (a^b) + (b^c) + (c^a);
}

vector<point> convex_hull(vector<point> p) {
	int n = p.size(), k = 0;
	if (n == 1) return p;
	vector<point> hull(2*n);

	sort(p.begin(), p.end());

	for(int i=0; i<n; i++) {
		// use <= when including collinear points
		while(k>=2 && (side(hull[k-2], hull[k-1], p[i]) < 0))
			k--;
		hull[k++] = p[i];
	}

	for(int i=n-2,t=k+1; i>=0; i--) {
		while(k>=t && (side(hull[k-2], hull[k-1], p[i]) < 0))
			k--;
		hull[k++] = p[i];
	}

	hull.resize(k-1);
	return hull;
}
\end{lstlisting}
\section{Graph Algorithms}
\subsection{Tarjan}
\begin{lstlisting}[language=C++]
#include <bits/stdc++.h>
using namespace std;

const int N = 2e5 + 5;
const int inf = 1791791791;

vector<int> conn[N];

// time complexity: O(V+E)
stack<int> ts;
int tme = 0, ncomp = 0, low[N], seen[N];
int comp[N]; // nodes in the same scc have the same color
int scc_dfs(int n) {
	seen[n] = low[n] = ++tme;
	ts.push(n);
	for (auto a : conn[n]) {
		if (seen[a] == 0)
			scc_dfs(a);
		low[n] = min(low[n], low[a]);
	}
	if (low[n] == seen[n]) {
		int node;
		do {
			node = ts.top(); ts.pop();
			comp[node] = ncomp;
			low[node] = inf;
		} while (n != node && ts.size());
		ncomp++;
	}
	return low[n];
}

int main() {
	int n, m; scanf("%d %d", &n, &m);
	while (m--) {
		int a, b; scanf("%d %d", &a, &b);	
		conn[a].push_back(b);
	}
	map<int, vector<int> > comps;
	for (int i=0;i<n;i++) {
		if (!seen[i]) scc_dfs(i);
		comps[comp[i]].push_back(i);
	}
	for (auto a : comps) {
		printf("%d: ", a.first);
		for (auto v : a.second)
			printf("%d ", v);
		printf("\n");
	}
}
\end{lstlisting}
\subsection{Lowest Common Ancestor}
\begin{lstlisting}[language=C++]
const int N = 1e6 + 5;
const int L = 20;

vector<int> adj[N];
int prof[N], p[N][L+5];

void dfs(int v, int h = 1) {
	prof[v] = h;
	if (h == 1) p[v][0] = v;
	for (auto u : adj[v])
		if (prof[u] == 0) {
			p[u][0] = v;
			dfs(u, h+1);
		}
}

void init(int n) {
	for (int i = 1; i <= L; i++)
		for (int j = 1; j < n; j++)
			p[j][i] = p[p[j][i-1]][i-1];
}

int lca(int u, int v) {
	if (prof[u] < prof[v]) swap(u, v);
	for (int i = L; i >= 0; i--)
		if (prof[p[u][i]] >= prof[v])
			u = p[u][i];
	for (int i = L; i >= 0; i--)
		if (p[u][i] != p[v][i]) {
			u = p[u][i];
			v = p[v][i];
		}
	while (u != v) {
		u = p[u][0];
		v = p[v][0];
	}
	return u;
}
\end{lstlisting}
\section{Flow}
\subsection{Dinitz' Algorithm}
\begin{lstlisting}[language=C++]
struct dinitz {
	struct edge {
		int from, to;
		ll c, f;
	};
	vector<edge> edges;
	vector<int> adj[N];

	void addEdge(int i, int j, ll c) {
		edges.push_back({i, j, c, 0}); adj[i].push_back(edges.size() - 1);
		edges.push_back({j, i, 0, 0}); adj[j].push_back(edges.size() - 1);
	}

	int turn, seen[N], dist[N], st[N];
	bool bfs (int s, int t) {
		seen[t] = ++turn;
		dist[t] = 0; 
		queue<int> q({t});
		while (q.size()) {
			int u = q.front(); q.pop();
			st[u] = 0;
			for (auto e : adj[u]) {
				int v = edges[e].to;
				if (seen[v] != turn && edges[e^1].c != edges[e^1].f) {
					seen[v] = turn;
					dist[v] = dist[u] + 1;
					q.push(v);
				}
			}
		}
		return seen[s] == turn;
	}

	ll dfs(int s, int t, ll f) {
		if (s == t || f == 0)
			return f;
		for (int &i = st[s]; i < adj[s].size(); i++) {
			int e = adj[s][i], v = edges[e].to;
			if (seen[v] == turn && dist[v] + 1 == dist[s] && edges[e].c > edges[e].f) {
				if (ll nf = dfs(v, t, min(f, edges[e].c - edges[e].f))) {
					edges[e].f += nf;
					edges[e^1].f -= nf;
					return nf;
				}
			}
		}
		return 0ll;
	}

	ll max_flow(int s, int t) {
		ll resp = 0ll;
		while (bfs(s, t))
			while (ll val = dfs(s, t, inf))
				resp += val;
		return resp;
	}
};
\end{lstlisting}
\subsection{Min Cost}
\begin{lstlisting}[language=C++]
typedef long long ll;
const ll inf = 1e12;

struct min_cost {
	struct edge {
		int from, to;
		ll cp, fl, cs;
	};
	vector<edge> edges;
	vector<int> adj[N];
	
	void addEdge(int i, int j, ll cp, ll cs) {
		edges.push_back({i, j, cp, 0, cs}); adj[i].push_back(edges.size() - 1);
		edges.push_back({j, i, 0, 0, -cs}); adj[j].push_back(edges.size() - 1);
	}

	ll seen[N], dist[N], pai[N], cost, flow;
	int turn;
	ll spfa(int s, int t) {
		turn++;
		queue<int> q; q.push(s);
		for (int i = 0; i < N; i++) dist[i] = inf;
		dist[s] = 0;
		seen[s] = turn;
		while (q.size()) {
			int u = q.front(); q.pop();
			seen[u] = 0;
			for (auto e : adj[u]) {
				int v = edges[e].to;
				if (edges[e].cp > edges[e].fl && dist[u] + edges[e].cs < dist[v]) {
					dist[v] = dist[u] + edges[e].cs;
					pai[v] = e ^ 1;
					if (seen[v] < turn) {
						seen[v] = turn;
						q.push(v);
					}
				}
			}
		}
		if (dist[t] == inf) return 0;
		ll nfl = inf;
		for (int u = t; u != s; u = edges[pai[u]].to)
			nfl = min(nfl, edges[pai[u] ^ 1].cp - edges[pai[u] ^ 1].fl);
		cost += dist[t] * nfl;
		for (int u = t; u != s; u = edges[pai[u]].to) {
			edges[pai[u]].fl -= nfl;
			edges[pai[u] ^ 1].fl += nfl;
		}
		return nfl;
	}

	void mncost(int s, int t) {
		cost = flow = 0;
		while (ll fl = spfa(s, t))
			flow += fl;
	}
};
\end{lstlisting}
\section{Data Structures}
\subsection{Trie}
\begin{lstlisting}[language=C++]
struct trie {
	struct node {
		int to[A], freq, end;
	};
	struct node t[N];
	int sz = 0;
	int offset = 'a';
	
	// init trie
	void init() {
		memset(t, 0, sizeof(struct node));
	}
	
	// insert string
	void insert(char *s, int p = 0) {
		t[p].freq++;
		if (*s == 0) {
			t[p].end++;
			return;
		}
		if (t[p].to[*s - offset] == 0)
			t[p].to[*s - offset] = ++sz;
		insert(s+1, t[p].to[*s - offset]);
	}

	// check if string is on trie
	int find(char *s, int p = 0) {
		if (*s == 0)
			return t[p].end;
		if (t[p].to[*s - offset] == 0)
			return false;
		return find(s+1, t[p].to[*s - offset]);
	}
	
	// count the number of strings that have this prefix
	int count(char *s, int p = 0) {
		if (*s == 0)
			return t[p].freq;
		if (t[p].to[*s - offset] == 0)
			return 0;
		return count(s+1, t[p].to[*s - offset]);
	}

	// erase a string
	int erase(char *s, int p = 0) {
		if (*s == 0 && t[p].end) {
			--t[p].end;
			return --t[p].freq;
		}
		if ((*s == 0 && t[p].end == 0) || t[p].to[*s - offset] == 0)
			return -1;
		int count = erase(s+1, t[p].to[*s - offset]);
		if (count == 0)
			t[p].to[*s - offset] = 0;
		if (count == -1)
			return -1;
		return --t[p].freq;
	}
};
\end{lstlisting}
\subsection{Binary Indexed Tree}
\begin{lstlisting}[language=C++]
int b[N];

int update(int p, int val, int n) {
	for(;p < n; p += p & -p) b[p] += val;
}

int getsum(int p) {
	int sum = 0;
	for(; p != 0; p -= p & -p) {
		sum += b[p];
	}
	return sum;
}
\end{lstlisting}
\subsection{Lazy Segment Tree}
\begin{lstlisting}[language=C++]
typedef long long ll;

const ll N = 1e5 + 5;
const ll inf = 1791791791;

struct seg_tree {
	ll seg[4*N];
	ll lazy[4*N];

	seg_tree() {
		memset(seg, 0, sizeof(seg));
		memset(lazy, 0, sizeof(lazy));
	}

	void do_lazy(ll root, ll left, ll right) {
		seg[root] += lazy[root];
		if (left != right) {
			lazy[2*root+1] += lazy[root];
			lazy[2*root+2] += lazy[root];
		}
		lazy[root] = 0;
	}

	// sum update
	ll update(ll l, ll r, ll val, ll left = 0, ll right = N-1, ll root = 0) {
		do_lazy(root, left, right);
		if (r < left || l > right) return seg[root];
		if (left >= l && right <= r) {
			lazy[root] += val;
			do_lazy(root, left, right);
			return seg[root];
		}
		ll update_left = update(l, r, val, left, (left+right)/2, 2*root+1);
		ll update_right = update(l, r, val, (left+right)/2+1, right, 2*root+2);
		return seg[root] = min(update_left, update_right);
	}

	ll query(ll l, ll r, ll left = 0, ll right = N-1, int root = 0) {
		do_lazy(root, left, right);
		if (r < left || l > right)
			return inf;
		if (left >= l && right <= r) return seg[root];
		ll query_left = query(l, r, left, (left+right)/2, 2*root+1);
		ll query_right = query(l, r, (left+right)/2+1, right, 2*root+2);
		return min(query_left, query_right);
	}
};
\end{lstlisting}
\subsection{Union Find}
\begin{lstlisting}[language=C++]
#include <bits/stdc++.h>
using namespace std;

const int N = 5e5 + 5;
int p[N], w[N];

int find(int x) {
	return p[x] = (x == p[x] ? x : find(p[x]));
}

void join(int a, int b) {
	if ((a = find(a)) == (b = find(b))) return;
	if (w[a] < w[b]) swap(a, b);
	w[a] += w[b];
	p[b] = a;
}

int main() {
	int n;
	scanf("%d", &n);
	for(int i=0;i<n;i++)
		w[p[i] = i] = 1;
	return 0;
}
\end{lstlisting}
\section{Mathematics}
\subsection{Matrix}
\begin{lstlisting}[language=C++]
template <int n> struct matrix {
	long long mat[n][n];
	matrix () {
		memset (mat, 0, sizeof (mat));
	}
	matrix (long long temp[n][n]) {
		memcpy (mat, temp, sizeof (mat));
	}
	void identity() {
		memset (mat, 0, sizeof (mat));
		for (int i=0;i<n;i++)
			mat[i][i] = 1;
	}
	matrix<n> mul (const matrix<n> &a, long long m) const {
		matrix<n> temp;
		for (int i=0; i<n; i++)
			for (int j=0; j<n; j++)
				for (int k=0; k<n; k++) {
					temp.mat[i][j] += (mat[i][k]*a.mat[k][j])%m;
					temp.mat[i][j] %= m;
				}
		return temp;
	}
	matrix<n> operator% (long long m) {
		matrix<n> temp(mat);
		for (int i=0; i<n; i++)
			for (int j=0; j<n; j++)
				temp.mat[i][j] %= m;
		return temp;
	}
	matrix<n> pow(long long e, long long m) {
		matrix<n> temp;
		if (e == 0) {
			temp.identity();
			return temp%m;
		}
		if (e == 1) {
			memcpy (temp.mat, mat, sizeof (temp.mat));
			return temp%m;
		}
		temp = pow(e/2, m);
		if (e % 2 == 0)
			return (temp.mul(temp, m))%m;
		else
			return (((temp.mul(temp, m))%m)*pow(1, m))%m;
	}
};
\end{lstlisting}
\subsection{Fast Fourier Transform}
\begin{lstlisting}[language=C++]
// This solves VFMUL on SPOJ
#include <bits/stdc++.h>
using namespace std;

#define PI 3.14159265359

const int N = 3e5 + 5;
typedef complex<double> base;

// p[0]*x^0 + p[1]*x + ...
void fft(vector<base> &p, bool inverse) {
	if (p.size() == 1)
		return;
	int n = p.size();

	vector<base> a[2];
	for (int i=0; i<n; i++) 
		a[i%2].push_back(p[i]);
	
	for (int i=0; i<2; i++)
		fft(a[i], inverse);

	double theta = (2*PI/n)*(inverse ? -1 : 1);
	base w(1), wn(cos(theta), sin(theta));
	for (int i=0; i<n/2; i++) {
		p[i] = (a[0][i] + w * a[1][i]) / (base)(inverse ? 2 : 1);
		p[i+n/2] = (a[0][i] - w * a[1][i]) / (base)(inverse ? 2 : 1);
		w *= wn;
	}
}

// c ends being a*b
void multiply(vector<int> &a, vector<int> &b, vector<int> &c) {
	vector<base> na(a.begin(), a.end()), nb(b.begin(), b.end());
	int n = 1;
	while (n < max(a.size(), b.size())) n <<= 1;
	n <<= 1;
	na.resize(n); nb.resize(n);
	
	fft(na, false); fft(nb, false);
	for (int i=0;i<n;i++) {
		na[i] *= nb[i];	
	}
	fft(na, true);
	
	c.resize(n);
	for (int i=0;i<n;i++)
		c[i] = (int)(na[i].real() + 0.5);
}

int main() {
	int t; scanf("%d", &t);
	while (t--) {
		char s1[N], s2[N];
		scanf("%s %s", s1, s2);
		int n1 = strlen(s1), n2 = strlen(s2);
		vector<int> a, b, c;

		for (int i=n1-1;i>=0;i--)
			a.push_back(s1[i]-'0');
		for (int i=n2-1;i>=0;i--)
			b.push_back(s2[i]-'0');
		multiply(a, b, c);
		
		c.resize(2*c.size());
		for (int i=0;i<c.size()-1;i++) {
			c[i+1] += c[i]/10;
			c[i] %= 10;
		}

		int found = 0;
		for (int i=c.size()-1;i>=0; i--) {
			if (c[i] != 0) found = 1;
			if (found) printf("%c", c[i] + '0');
		}
		if (!found) printf("0");
		printf("\n");
	}
    return 0;
}

\end{lstlisting}
\subsection{Extended Euclidean Algorithm}
\begin{lstlisting}[language=C++]
// This solves 10104 on UVa
#include <bits/stdc++.h>
using namespace std;

typedef long long ll;

ll ext(ll a, ll b, ll &x, ll &y) {
	if (a == 0) {
		x = 0;
		y = 1;
		return b;
	}
	ll x1, y1;
	ll gcd = ext(b%a, a, x1, y1);

	x = y1 - (b/a)*x1;
	y = x1;

	return gcd;
}

int main() {
	ll a, b;
	while (scanf("%lld %lld", &a, &b) != EOF) {
		ll x, y;
		ll gcd = ext(a, b, x, y);
		if (a == b && x > y) swap(x, y);
		printf("%lld %lld %lld\n", x, y, gcd);	
	}
    return 0;
}

\end{lstlisting}
\subsection{Rabin-Miller Primality Test}
\begin{lstlisting}[language=C++]
long long llrand(long long mn, long long mx) {
	long long p = rand();
	p <<= 32ll;
	p += rand();
	return p%(mx-mn+1ll)+mn;
}

long long mul_mod(long long a, long long b, long long m) {
	long long x = 0, y = a%m;
	while (b) {
		if (b % 2)
			x = (x+y)%m;
		y = (2*y)%m;
		b >>= 1;
	}
	return x%m;
}

long long exp_mod(long long e, long long n, long long m) {
	if (n == 0)
		return 1ll;
	long long temp = exp_mod(e, n/2, m);
	if (n & 1)
		return mul_mod(mul_mod(temp, temp, m), e, m);
	else
		return mul_mod(temp, temp, m);
}

// complexity: O(t*log2^3(p))
bool isProbablyPrime(long long p, long long t=64) {
	if (p <= 1) return false;
	if (p <= 3) return true;
	srand(time(NULL));
	long long r = 0, d = p-1;
	while (d % 2 == 0) {
		r++;
		d >>= 1;
	}
	while (t--) {
		long long a = llrand(2, p-2);
		a = exp_mod(a, d, p);
		if (a == 1 || a == p-1) continue;
		for (int i=0; i<r-1; i++) {
			a = mul_mod(a, a, p);
			if (a == 1) return false;
			if (a == p-1) break;
		}
		if (a != p-1) return false;
	}
	return true;
}
\end{lstlisting}
\section{Strings}
\subsection{Z function}
\begin{lstlisting}[language=C++]
int z[N];

void Z(string s) {
	int n = s.size();
	int m = -1;
	for (int i = 1; i < n; i++) {
		z[i] = 0;
		if (m != -1 && m + z[m] >= i)
			z[i] = min(m + z[m] - i, z[i-m]);
		while (i + z[i] < n && s[i+z[i]] == s[z[i]])
			z[i]++;
		if (m == -1 || i + z[i] > m + z[m])
			m = i;
	}
}
\end{lstlisting}
\subsection{Knuth–Morris–Pratt Algorithm}
\begin{lstlisting}[language=C++]
int kmp[N];

void build(string p) {
	int n = p.size(), k = -1;
	kmp[0] = k;
	for (int i = 1; i < n+1; i++) {
		while (k >= 0 && p[k] != p[i-1]) k = kmp[k];
		kmp[i] = ++k;
	}
}

vector<int> match(string p, string s) {
	int n = s.size(), m = p.size(), j = 0;
	vector<int> matches;
	for (int i = 1; i < n+1; i++) {
		while (j >= 0 && p[j] != s[i-1]) j = kmp[j];
		if (++j == m) {
			matches.push_back(i-j+1);
			j = kmp[j];
		}
	}
	return matches;
}
\end{lstlisting}
\section{Miscellaneous}
\subsection{vim settings}
\begin{lstlisting}[language=]
set ai si noet ts=4 sw=4 sta sm nu rnu
inoremap <NL> <ESC>o
nnoremap <NL> o
inoremap <C-up> <C-o>:m-2<CR>
inoremap <C-down> <C-o>:m+1<CR>
nnoremap <C-up> :m-2<CR>
nnoremap <C-down> :m+1<CR>
vnoremap <C-up> :m-2<CR>gv
vnoremap <C-down> :m'>+1<CR>gv
syntax on
colors evening
highlight Normal ctermbg=none "No background
highlight nonText ctermbg=none
\end{lstlisting}

%\end{multicols}
\end{document}

