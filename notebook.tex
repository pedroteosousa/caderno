\documentclass{article}
\usepackage[utf8]{inputenc}

\title{ICPC Notebook}
\author{pedroteosousa }
\date{}

\usepackage{listings}
\usepackage{xcolor}

\lstset{
	tabsize=4,
	showstringspaces=false,
	commentstyle=\color{black},
	keywordstyle=\color{blue},
	stringstyle=\color{red}
}

\begin{document}

\maketitle
\tableofcontents

\section{Geometry}
\subsection{Miscellaneous Geometry}
\begin{lstlisting}[language=C++]
#include <bits/stdc++.h>
using namespace std;

typedef double type;
double EPS = 1e-12;

struct point {
    type x, y;
    point(type xp = 0.0, type yp = 0.0) {
        x = xp;
        y = yp;
    }
    point(const point &p) {
        x = p.x;
        y = p.y;
    }
    point operator+ (const point &p) const { return point(x+p.x, y+p.y); }
    point operator- (const point &p) const { return point(x-p.x, y-p.y); }
    point operator* (type c) { return point(c*x, c*y); }
    point operator/ (type c) { return point(x/c, y/c); }

    bool operator<(const point &p) {
		return x < p.x || x == p.x && y < p.y;
	}
};

type dot(point p, point q)  { return p.x*q.x+p.y*q.y; }
type dist(point p, point q)  { return sqrt(dot(p-q,p-q)); }
type cross(point p, point q)  { return p.x*q.y-p.y*q.x; }

point projectInLine(point c, point a, point b) {
	return a + (b-a)*dot(c-a, b-a)/dot(b-a, b-a);
}

point projectInSegment(point c, point a, point b) {
    point lineP = projectInLine(c, a, b);
    type maxDist = max(dist(a, lineP), dist(b, lineP));
    if (maxDist > dist(a, b)) {
        if (dist(a, c) > dist(b, c)) return b;
        else return a;
    }
    else return lineP;
}

int main() {
	point a(0, 0), b(1, 1), c(1, 0);
	c = projectInSegment(c, a, b);
	printf("%lf %lf\n", c.x, c.y);
}
\end{lstlisting}
\subsection{Convex Hull}
\begin{lstlisting}[language=C++]
// This solves problem E on codeforces gym 101484
#include <bits/stdc++.h>
using namespace std;

typedef long long type;
double EPS = 1e-12;

struct point {
    type x, y;
    point(type xp = 0, type yp = 0) {
        x = xp;
        y = yp;
    }

    bool operator<(const point &p) const {
		return x < p.x || x == p.x && y < p.y;
	}
};

type cross(point p, point q)  { return p.x*q.y-p.y*q.x; }

double side(point a, point b, point c) {
    return cross(a, b) + cross(b, c) + cross(c, a);
}

vector<point> convex_hull(vector<point> p) {
    int n = p.size(), k = 0;
    if (n == 1) return p;
    vector<point> hull(2*n);

    sort(p.begin(), p.end());

    for(int i=0; i<n; i++) {
		// use <= when including collinear points
        while(k>=2 && (side(hull[k-2], hull[k-1], p[i]) < 0)) k--;
        hull[k++] = p[i];
    }

    for(int i=n-2,t=k+1; i>=0; i--) {
        while(k>=t && (side(hull[k-2], hull[k-1], p[i]) < 0)) k--;
        hull[k++] = p[i];
    }

    hull.resize(k-1);
    return hull;
}

set<point> v1, v2;
int main() {
	int n, m; scanf("%d %d", &n, &m);
	vector<point> h;
	for (int i=0;i<n;i++) {
		int a, b; scanf("%d %d", &a, &b);
		point c = point(a, b);
		v1.insert(c); h.push_back(c);
	}
	for (int i=0;i<m;i++) {
		int a, b; scanf("%d %d", &a, &b);
		point c = point(a, b);
		v2.insert(c); h.push_back(c);
	}
	h = convex_hull(h);
	if (v1.find(h[0]) != v1.end()) {
		for (int i=0;i<h.size();i++)
			if (v2.find(h[i]) != v2.end()) {
				printf("NO\n"); return 0;
			}
	}
	else {
		for (int i=0;i<h.size();i++)
			if (v1.find(h[i]) != v1.end()) {
				printf("NO\n"); return 0;
			}
	}
	printf("YES\n");
}
\end{lstlisting}
\section{Graph Algorithms}
\subsection{Tarjan}
\begin{lstlisting}[language=C++]
#include <bits/stdc++.h>
using namespace std;

const int N = 2e5 + 5;
const int inf = 1791791791;

vector<int> conn[N];

// time complexity: O(V+E)
stack<int> ts;
int tme = 0, ncomp = 0, low[N], seen[N];
int comp[N]; // nodes in the same scc have the same color
int scc_dfs(int n) {
	seen[n] = low[n] = ++tme;
	ts.push(n);
	for (auto a : conn[n]) {
		if (seen[a] == 0)
			scc_dfs(a);
		low[n] = min(low[n], low[a]);
	}
	if (low[n] == seen[n]) {
		int node;
		do {
			node = ts.top(); ts.pop();
			comp[node] = ncomp;
			low[node] = inf;
		} while (n != node && ts.size());
		ncomp++;
	}
	return low[n];
}

int main() {
	int n, m; scanf("%d %d", &n, &m);
	while (m--) {
		int a, b; scanf("%d %d", &a, &b);	
		conn[a].push_back(b);
	}
	map<int, vector<int> > comps;
	for (int i=0;i<n;i++) {
		if (!seen[i]) scc_dfs(i);
		comps[comp[i]].push_back(i);
	}
	for (auto a : comps) {
		printf("%d: ", a.first);
		for (auto v : a.second)
			printf("%d ", v);
		printf("\n");
	}
}
\end{lstlisting}
\section{Data Structures}
\subsection{Trie}
\begin{lstlisting}[language=C++]
const int A = 26;

typedef struct trie {
	struct node {
		int to[A], freq, end;
	};
	struct node t[N];
	int sz = 0;
	int offset = 'a';
	
	// init trie
	void init() {
		memset(t, 0, sizeof(struct node));
	}
	// insert string
	void insert(char *s, int p = 0) {
		t[p].freq++;
		if (*s == 0) {
			t[p].end++;
			return;
		}
		if (t[p].to[*s - offset] == 0)
			t[p].to[*s - offset] = ++sz;
		insert(s+1, t[p].to[*s - offset]);
	}

	// check if string is on trie
	int find(char *s, int p = 0) {
		if (*s == 0)
			return t[p].end;
		if (t[p].to[*s - offset] == 0)
			return false;
		return find(s+1, t[p].to[*s - offset]);
	}
	
	// count the number of strings that have this prefix
	int count(char *s, int p = 0) {
		if (*s == 0)
			return t[p].freq;
		if (t[p].to[*s - offset] == 0)
			return 0;
		return count(s+1, t[p].to[*s - offset]);
	}

	// erase a string
	int erase(char *s, int p = 0) {
		if (*s == 0 && t[p].end) {
			--t[p].end;
			return --t[p].freq;
		}
		if ((*s == 0 && t[p].end == 0) || t[p].to[*s - offset] == 0)
			return -1;
		int count = erase(s+1, t[p].to[*s - offset]);
		if (count == 0)
			t[p].to[*s - offset] = 0;
		if (count == -1)
			return -1;
		return --t[p].freq;
	}
} trie;
\end{lstlisting}
\subsection{Binary Indexed Tree}
\begin{lstlisting}[language=C++]
int b[N];

int update(int p, int val, int n) {
	for(;p < n; p += p & -p) b[p] += val;
}

int getsum(int p) {
	int sum = 0;
	for(; p != 0; p -= p & -p) {
		sum += b[p];
	}
	return sum;
}
\end{lstlisting}
\subsection{Lazy Segment Tree}
\begin{lstlisting}[language=C++]
// This solves HORRIBLE on SPOJ
#include <bits/stdc++.h>
using namespace std;

typedef long long lli;

const lli N = 1e5 + 5;
const lli inf = 1791791791;

/* type: 0 = min
   		 1 = max
   		 2 = sum */
template <int type> struct seg_tree {
	lli seg[4*N];
	lli lazy[4*N];
	
	seg_tree() {
		memset(seg, 0, sizeof(seg));
		memset(lazy, 0, sizeof(lazy));
	}

	void do_lazy(lli root, lli ll, lli rl) {
		if (type == 2)
			seg[root] += (rl-ll+1)*lazy[root];
		else
			seg[root] += lazy[root];
		if (ll != rl) {
			lazy[2*root+1] += lazy[root];
			lazy[2*root+2] += lazy[root];
		}
		lazy[root] = 0;
	}

	// sum update
	lli update(lli l, lli r, lli val, lli ll = 0, lli rl = N-1, lli root = 0) {
		do_lazy(root, ll, rl);
		if (r < ll || l > rl) return seg[root];
		if (ll >= l && rl <= r) {
			lazy[root] += val;
			do_lazy(root, ll, rl);
			return seg[root];
		}
		lli update_left = update(l, r, val, ll, (ll+rl)/2, 2*root+1);
		lli update_right = update(l, r, val, (ll+rl)/2+1, rl, 2*root+2);
		if (type == 0)	
			return seg[root] = min(update_left, update_right);
		if (type == 1)	
			return seg[root] = max(update_left, update_right);
		if (type == 2)
			return seg[root] = update_left + update_right;
	}

	lli query(lli l, lli r, lli ll = 0, lli rl = N-1, int root = 0) {
		do_lazy(root, ll, rl);
		if (r < ll || l > rl) {
			if (type == 0)
				return inf;
			if (type == 1)
				return -inf;
			if (type == 2)
				return 0;
		}
		if (ll >= l && rl <= r) return seg[root];
		lli query_left = query(l, r, ll, (ll+rl)/2, 2*root+1);
		lli query_right = query(l, r, (ll+rl)/2+1, rl, 2*root+2);
		if (type == 0)
			return min(query_left, query_right);
		if (type == 1)
			return max(query_left, query_right);
		if (type == 2)
			return query_left + query_right;
	}
};

int main() {
	int t; scanf("%d", &t);
	while (t--) {
		int n, c; scanf("%d %d", &n, &c);
		seg_tree <2> t;
		while (c--) {
			int op, l, r;
			scanf("%d %d %d", &op, &l, &r);
			l--; r--;
			if (op == 0) {
				int v; scanf("%d", &v);
				t.update(l, r, v);
			}
			else
				printf("%lld\n", t.query(l, r));
		}
	}
}
\end{lstlisting}
\subsection{Union Find}
\begin{lstlisting}[language=C++]
#include <bits/stdc++.h>
using namespace std;

const int N = 5e5 + 5;
int p[N], w[N];

int find(int x) {
	return p[x] = (x == p[x] ? x : find(p[x]));
}

void join(int a, int b) {
	if ((a = find(a)) == (b = find(b))) return;
	if (w[a] < w[b]) swap(a, b);
	w[a] += w[b];
	p[b] = a;
}

int main() {
	int n;
	scanf("%d", &n);
	for(int i=0;i<n;i++)
		w[p[i] = i] = 1;
	return 0;
}
\end{lstlisting}
\section{Mathematics}
\subsection{Matrix}
\begin{lstlisting}[language=C++]
// This solves problem MAIN74 on SPOJ
#include <bits/stdc++.h>
using namespace std;

const int mod = 1e9+7;

template <int n> struct matrix {
	long long mat[n][n];
	matrix () {
		memset (mat, 0, sizeof (mat));
	}
	matrix (long long temp[n][n]) {
		memcpy (mat, temp, sizeof (mat));
	}
	void identity() {
		memset (mat, 0, sizeof (mat));
		for (int i=0;i<n;i++)
			mat[i][i] = 1;
	}
	matrix<n> operator* (const matrix<n> &a) const {
		matrix<n> temp;
		for (int i=0; i<n; i++)
			for (int j=0; j<n; j++)
				for (int k=0; k<n; k++)
					temp.mat[i][j] += mat[i][k]*a.mat[k][j];
		return temp;
	}
	matrix<n> operator% (long long m) {
		matrix<n> temp(mat);
		for (int i=0; i<n; i++)
			for (int j=0; j<n; j++)
				temp.mat[i][j] %= m;
		return temp;
	}
	matrix<n> pow(long long e, long long m) {
		matrix<n> temp;
		if (e == 0) {
			temp.identity();
			return temp%m;
		}
		if (e == 1) {
			memcpy (temp.mat, mat, sizeof (temp.mat));
			return temp%m;
		}
		temp = pow(e/2, m);
		if (e % 2 == 0)
			return (temp*temp)%m;
		else
			return (((temp*temp)%m)*pow(1, m))%m;
	}
};

int main() {
	int t;
	scanf("%d", &t);
	while (t--) {
		long long n;
		scanf("%lld", &n);
		matrix<2> m;
		long long temp[2][2] = {{1, 1}, {1, 0}};
		memcpy (m.mat, temp, sizeof (m.mat));
		m = m.pow(n+2, mod);
		if (n == 0) m.mat[0][0] = 0;
		if (n == 1) m.mat[0][0] = 2;
		printf("%lld\n", m.mat[0][0]);
	}
    return 0;
}

\end{lstlisting}
\subsection{Fast Fourier Transform}
\begin{lstlisting}[language=C++]
// This solves VFMUL on SPOJ
#include <bits/stdc++.h>
using namespace std;

#define PI 3.14159265359

const int N = 3e5 + 5;
typedef complex<double> base;

// p[0]*x^0 + p[1]*x + ...
void fft(vector<base> &p, bool inverse) {
	if (p.size() == 1)
		return;
	int n = p.size();

	vector<base> a[2];
	for (int i=0; i<n; i++) 
		a[i%2].push_back(p[i]);
	
	for (int i=0; i<2; i++)
		fft(a[i], inverse);

	double theta = (2*PI/n)*(inverse ? -1 : 1);
	base w(1), wn(cos(theta), sin(theta));
	for (int i=0; i<n/2; i++) {
		p[i] = (a[0][i] + w * a[1][i]) / (base)(inverse ? 2 : 1);
		p[i+n/2] = (a[0][i] - w * a[1][i]) / (base)(inverse ? 2 : 1);
		w *= wn;
	}
}

// c ends being a*b
void multiply(vector<int> &a, vector<int> &b, vector<int> &c) {
	vector<base> na(a.begin(), a.end()), nb(b.begin(), b.end());
	int n = 1;
	while (n < max(a.size(), b.size())) n <<= 1;
	n <<= 1;
	na.resize(n); nb.resize(n);
	
	fft(na, false); fft(nb, false);
	for (int i=0;i<n;i++) {
		na[i] *= nb[i];	
	}
	fft(na, true);
	
	c.resize(n);
	for (int i=0;i<n;i++)
		c[i] = (int)(na[i].real() + 0.5);
}

int main() {
	int t; scanf("%d", &t);
	while (t--) {
		char s1[N], s2[N];
		scanf("%s %s", s1, s2);
		int n1 = strlen(s1), n2 = strlen(s2);
		vector<int> a, b, c;

		for (int i=n1-1;i>=0;i--)
			a.push_back(s1[i]-'0');
		for (int i=n2-1;i>=0;i--)
			b.push_back(s2[i]-'0');
		multiply(a, b, c);
		
		c.resize(2*c.size());
		for (int i=0;i<c.size()-1;i++) {
			c[i+1] += c[i]/10;
			c[i] %= 10;
		}

		int found = 0;
		for (int i=c.size()-1;i>=0; i--) {
			if (c[i] != 0) found = 1;
			if (found) printf("%c", c[i] + '0');
		}
		if (!found) printf("0");
		printf("\n");
	}
    return 0;
}

\end{lstlisting}
\subsection{Extended Euclidean Algorithm}
\begin{lstlisting}[language=C++]
// This solves 10104 on UVa
#include <bits/stdc++.h>
using namespace std;

typedef long long ll;

ll ext(ll a, ll b, ll &x, ll &y) {
	if (a == 0) {
		x = 0;
		y = 1;
		return b;
	}
	ll x1, y1;
	ll gcd = ext(b%a, a, x1, y1);

	x = y1 - (b/a)*x1;
	y = x1;

	return gcd;
}

int main() {
	ll a, b;
	while (scanf("%lld %lld", &a, &b) != EOF) {
		ll x, y;
		ll gcd = ext(a, b, x, y);
		if (a == b && x > y) swap(x, y);
		printf("%lld %lld %lld\n", x, y, gcd);	
	}
    return 0;
}

\end{lstlisting}
\subsection{Rabin-Miller Primality Test}
\begin{lstlisting}[language=C++]
// This (probably) solves PON on SPOJ
#include <bits/stdc++.h>
using namespace std;

long long llrand(long long mn, long long mx) {
	long long p = rand();
	p <<= 32ll;
	p += rand();
	return p%(mx-mn+1ll)+mn;
}

long long mul_mod(long long a, long long b, long long m) {
	long long x = 0, y = a%m;
	while (b) {
		if (b % 2)
			x = (x+y)%m;
		y = (2*y)%m;
		b >>= 1;
	}
	return x%m;
}

long long exp_mod(long long e, long long n, long long m) {
	if (n == 0)
		return 1ll;
	long long temp = exp_mod(e, n/2, m);
	if (n & 1)
		return mul_mod(mul_mod(temp, temp, m), e, m);
	else
		return mul_mod(temp, temp, m);
}

// complexity: O(t*log2^3(p))
bool isProbablyPrime(long long p, long long t=64) {
	if (p <= 1) return false;
	if (p <= 3) return true;
	srand(time(NULL));
	long long r = 0, d = p-1;
	while (d % 2 == 0) {
		r++;
		d >>= 1;
	}
	while (t--) {
		long long a = llrand(2, p-2);
		a = exp_mod(a, d, p);
		if (a == 1 || a == p-1) continue;
		for (int i=0; i<r-1; i++) {
			a = mul_mod(a, a, p);
			if (a == 1) return false;
			if (a == p-1) break;
		}
		if (a != p-1) return false;
	}
	return true;
}

int main() {
	int t; scanf("%d", &t);
	while (t--) {
		long long p; scanf("%lld", &p);
		if (isProbablyPrime(p)) printf("YES\n");
		else printf("NO\n");
	}
    return 0;
}

\end{lstlisting}
\section{Miscellaneous}
\subsection{vim settings}
\begin{lstlisting}[language=]
set ai si noet ts=4 sw=4 sta sm nu rnu
inoremap <NL> <ESC>o
nnoremap <NL> o
inoremap <C-up> <C-o>:m-2<CR>
inoremap <C-down> <C-o>:m+1<CR>
nnoremap <C-up> :m-2<CR>
nnoremap <C-down> :m+1<CR>
vnoremap <C-up> :m-2<CR>gv
vnoremap <C-down> :m'>+1<CR>gv
syntax on
colors evening
highlight Normal ctermbg=none "No background
highlight nonText ctermbg=none
\end{lstlisting}

\end{document}

