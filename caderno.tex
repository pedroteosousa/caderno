\documentclass{article}
\usepackage[utf8]{inputenc}

\title{Caderno}
\author{pedroteosousa }
\date{}

\usepackage{listings}
\usepackage{xcolor}

\lstset{
	tabsize=4,
	showstringspaces=false,
	commentstyle=\color{black},
	keywordstyle=\color{blue},
	stringstyle=\color{red}
}

\begin{document}

\maketitle
\tableofcontents

\section{Geometry}
\subsection{Point Struct}
\begin{lstlisting}[language=C++]
typedef long long type;

double EPS = 1e-12;

struct point {
    type x, y;
    point(type xp = 0.0, type yp = 0.0) {
        x = xp;
        y = yp;
    }
    point(const fpoint &p) {
        x = p.x;
        y = p.y;
    }
    point operator+ (const point &p) const {return point(x+p.x, y+p.y);}
    point operator- (const point &p) const {return point(x-p.x, y-p.y);}
    point operator* (type c) {return point(c*x, c*y);}
    point operator/ (type c) {return point(x/c, y/c);}

    bool operator<(const point &p) {return x < p.x || x == p.x && y < p.y;}
};

type dot(point p, point q) {return p.x*q.x+p.y*q.y;}
type dist(point p, point q) {return sqrt(dot(p-q,p-q));}
type cross(point p, point q) {return p.x*q.y-p.y*q.x;}

point projectInLine(point c, point a, point b) {
	return a + (b-a)*dot(c-a, b-a)/dot(b-a, b-a);
}
point projectInSegment(point c, point a, point b) {
    point lineP = projectInLine(c, a, b);
    type maxDist = max(dist(a, lineP), dist(b, lineP));
    if (maxDist > dist(a, b)) {
        if (dist(a, c) > dist(b, c)) return b;
        else return a;
    }
    else return lineP;
}\end{lstlisting}
\subsection{Convex Hull}
\begin{lstlisting}[language=C++]
double side(point a, point b, point c) {
    return cross(a, b) + cross(b, c) + cross(c, a);
}

vector<point> convex_hull(vector<point> p) {
    int n = p.size(), k = 0;
    if (n == 1) return p;
    vector<point> hull(2*n);

    sort(p.begin(), p.end());

    for(int i=0; i<n; i++) {
        while(k>=2 && (side(hull[k-2], hull[k-1], p[i]) <= 0)) k--;
        hull[k++] = p[i];
    }

    for(int i=n-2,t=k+1; i>=0; i--) {
        while(k>=t && (side(hull[k-2], hull[k-1], p[i]) <= 0)) k--;
        hull[k++] = p[i];
    }

    hull.resize(k-1);
    return hull;
}\end{lstlisting}
\section{Data Structures}
\subsection{Trie}
\begin{lstlisting}[language=C++]
const int A = 26;

typedef struct trie {
	struct node {
		int to[A], freq, end;
	};
	struct node t[N];
	int sz = 0;
	int offset = 'a';
	
	// init trie
	void init() {
		memset(t, 0, sizeof(struct node));
	}
	// insert string
	void insert(char *s, int p = 0) {
		t[p].freq++;
		if (*s == 0) {
			t[p].end++;
			return;
		}
		if (t[p].to[*s - offset] == 0)
			t[p].to[*s - offset] = ++sz;
		insert(s+1, t[p].to[*s - offset]);
	}

	// check if string is on trie
	int find(char *s, int p = 0) {
		if (*s == 0)
			return t[p].end;
		if (t[p].to[*s - offset] == 0)
			return false;
		return find(s+1, t[p].to[*s - offset]);
	}
	
	// count the number of strings that have this prefix
	int count(char *s, int p = 0) {
		if (*s == 0)
			return t[p].freq;
		if (t[p].to[*s - offset] == 0)
			return 0;
		return count(s+1, t[p].to[*s - offset]);
	}

	// erase a string
	int erase(char *s, int p = 0) {
		if (*s == 0 && t[p].end) {
			--t[p].end;
			return --t[p].freq;
		}
		if ((*s == 0 && t[p].end == 0) || t[p].to[*s - offset] == 0)
			return -1;
		int count = erase(s+1, t[p].to[*s - offset]);
		if (count == 0)
			t[p].to[*s - offset] = 0;
		if (count == -1)
			return -1;
		return --t[p].freq;
	}
} trie;
\end{lstlisting}
\subsection{BIT}
\begin{lstlisting}[language=C++]
int b[N];

int update(int p, int val, int n) {
	for(;p < n; p += p & -p) b[p] += val;
}

int getsum(int p) {
	int sum = 0;
	for(; p != 0; p -= p & -p) {
		sum += b[p];
	}
	return sum;
}
\end{lstlisting}
\subsection{Recursive Segment Tree}
\begin{lstlisting}[language=C++]
int t[N<<1];

void build(int n) {
	for(int i = n-1; i > 0; i--) t[i] = min(t[i<<1], t[i<<1|1]);
}

void modify(int pos, int val, int n) {
	for(t[pos += n] = val; pos != 1; pos>>=1)
		t[pos>>1] = min(t[pos], t[pos^1]);
}

int query(int l, int r, int n) { // [l, r)
	int resp = 1000000007;
	for(l += n, r += n; l < r; l >>= 1, r >>= 1) {
		if (l&1) resp = min(resp, t[l++]);
		if (r&1) resp = min(resp, t[--r]);
	}
	return resp;
}
\end{lstlisting}
\subsection{Lazy Segment Tree}
\begin{lstlisting}[language=C++]
int seg[4*N];
int lazy[4*N];

void do_lazy(int root, int ll, int rl) {
	seg[root] += lazy[root];
	if (ll != rl) {
		lazy[2*root+1] += lazy[root];
		lazy[2*root+2] += lazy[root];
	}
	lazy[root] = 0;
}

int update(int root, int ll, int rl, int l, int r, int val) {
	do_lazy(root, ll, rl);
	if (r < ll || l > rl) return seg[root];
	if (ll >= l && rl <= r) {
		lazy[root] += val;
		do_lazy(root, ll, rl);
		return seg[root];
	}
	int update_left = update(2*root+1, ll, (ll+rl)/2, l, r, val);
	int update_right = update(2*root+2, (ll+rl)/2+1, rl, l, r, val);
	return seg[root] = min(update_left, update_right);
}

int query(int root, int ll, int rl, int l, int r) {
	do_lazy(root, ll, rl);
	if (r < ll || l > rl) return inf;
	if (ll >= l && rl <= r) return seg[root];
	int query_left = query(2*root+1, ll, (ll+rl)/2, l, r);
	int query_right = query(2*root+2, (ll+rl)/2+1, rl, l, r);
	return min(query_left, query_right); 
}
\end{lstlisting}
\section{Mathematics}
\subsection{Matrix}
\begin{lstlisting}[language=C++]
// This code solves problem MAIN74 on SPOJ

#include <bits/stdc++.h>

using namespace std;

const int mod = 1e9+7;

template <int n> struct matrix {
	long long mat[n][n];
	matrix () {
		memset (mat, 0, sizeof (mat));
	}
	matrix (long long temp[n][n]) {
		memcpy (mat, temp, sizeof (mat));
	}
	void identity() {
		memset (mat, 0, sizeof (mat));
		for (int i=0;i<n;i++)
			mat[i][i] = 1;
	}
	matrix<n> operator* (const matrix<n> &a) const {
		matrix<n> temp;
		for (int i=0; i<n; i++)
			for (int j=0; j<n; j++)
				for (int k=0; k<n; k++)
					temp.mat[i][j] += mat[i][k]*a.mat[k][j];
		return temp;
	}
	matrix<n> operator% (long long m) {
		matrix<n> temp(mat);
		for (int i=0; i<n; i++)
			for (int j=0; j<n; j++)
				temp.mat[i][j] %= m;
		return temp;
	}
	matrix<n> pow(long long e, long long m) {
		matrix<n> temp;
		if (e == 0) {
			temp.identity();
			return temp%m;
		}
		if (e == 1) {
			memcpy (temp.mat, mat, sizeof (temp.mat));
			return temp%m;
		}
		temp = pow(e/2, m);
		if (e % 2 == 0)
			return (temp*temp)%m;
		else
			return (((temp*temp)%m)*pow(1, m))%m;
	}
};

int main() {
	int t;
	scanf("%d", &t);
	while (t--) {
		long long n;
		scanf("%lld", &n);
		matrix<2> m;
		memcpy (m.mat, (const long long [][2]){{1, 1}, {1, 0}}, sizeof (m.mat));
		m = m.pow(n+2, mod);
		if (n == 0) m.mat[0][0] = 0;
		if (n == 1) m.mat[0][0] = 2;
		printf("%lld\n", m.mat[0][0]);
	}
    return 0;
}

\end{lstlisting}

\end{document}

